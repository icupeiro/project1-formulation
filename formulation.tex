% This conference manuscript template is prepared for:   
% Kim Stockment, Conference Coordinator, Ray W. Herrick Laboratories, Purdue University, West Lafayette, IN, USA. 
% Latest revision = 2016-02-23 

\documentclass[10pt]{extarticle}
% Miktex users: require l3 packages
%             : optional 3 packages
\usepackage{amssymb,amsmath,multicol,titlesec,apacite,booktabs,tabto,url, multirow}
%\usepackage[round]{natbib}  

\usepackage[hmargin = 1in, vmargin = 1in]{geometry}

\renewcommand{\refname}{References}
    
%\usepackage[MnSymbol]{mathspec}
%\setallmainfonts{Times New Roman}
% Mathspec requires the XeTeX compiler. 

\usepackage[parfill]{parskip}
\usepackage{graphicx}
\usepackage[font=bf]{caption}

\usepackage{titlesec} 
%\titleformat{command}[shape]{format}{label}{sep}{before}[after]
\titleformat{\section}[hang]{\fontsize{12pt}{1em}\selectfont \bfseries}{\thesection. }{0pt}{\centering \MakeUppercase}
\titleformat{\subsection}[hang]{\fontsize{11pt}{1em}\selectfont \bfseries}{\thesubsection}{5pt}{}
\titleformat{\subsubsection}[runin]{}{\thesubsubsection}{5pt}{} 
%\titlespacing{command}{left}{beforesep}{aftersep}[right]
\titlespacing{\section}{0pt}{10pt}{10pt}
\titlespacing{\subsection}{0pt}{10pt}{0pt}
\titlespacing{\subsubsection}{0pt}{10pt}{0pt}	
%\setlength{\bibsep}{3pt}
%\pagenumbering{\gobble}
%\setlength{\hyphenpenalty}{1000}
%\setlength{\exhyphenpenalty}{1000}

\usepackage{fancyhdr}
\pagestyle{fancy}
\fancyhead[R]{\fontsize{12pt}{1em}\selectfont {{\textbf{Page \thepage}}}}
\fancyhead[L]{}
\fancyfoot[C]{IBPSA Project 1, September 2018}
\renewcommand{\headrulewidth}{0pt} % Turn off the bar

\newcommand\thefontsize[1]{{#1 The current font size is: \f@size pt\par}}

\begin{document}
	
\begin{center}
\vspace{0.2in}
\noindent{\fontsize{14pt}{1em}\selectfont \textbf{MPC formulation description template}}\\[14pt]

{\fontsize{11pt}{1.2em}\selectfont
Iago Cupeiro Figueroa\textsuperscript{1}, J\'an Drgo\v na\textsuperscript{2}
\\[11pt]

KU Leuven\\
Department of Mechanical Engineering - TME Division\\
The Sysi's \\[11pt]

iago.cupeirofigueroa@kuleuven.be\textsuperscript{1}, jan.drgona@kuleuven.be\textsuperscript{2} \\
}
\end{center}

\vspace{0.5cm}


\section*{ABSTRACT}

This document contains the MPC formulation template within the frame of the IBPSA Project 1.
The proposed template has to adapt all possible formulations for the different types of model and key performance indicators (KPIs) in consideration.
It has to be general.
To this end, we choose the control engineering notation as our starting point,
since it considers the formulation in a general way from an abstract point of view.
From here, we aim to move to a more physical formulation based on the particularities and specific characteristics of the building sector.

%\section*{NOMENCLATURE}
%
%\begin{samepage}
%	$x$\tabto{1.0in}Vector of states	\\
%	$u$\tabto{1.0in}Vector of inputs	\\
%	$y$\tabto{1.0in}Vector of outputs\\	
%	$d$\tabto{1.0in}Vector of disturbances\ \\
%	$N$\tabto{1.0in}Prediction horizon
%	
%\end{samepage}
%\begin{samepage}
%	\textbf{Subscript}\\
%	$k$\tabto{1.0in}Considered time-step index
%\end{samepage}


\section{CONTROL ENGINEERING NOTATION}

General MPC formulation can be cast as a following optimal control problem (OCP) in discrete time
\begin{subequations}
	\label{eq:mpc_general_formal}
	\begin{align}
	\min_{u_0, \ldots, u_{N-1}} & \ell_N(x_N) + \sum_{k=0}^{N-1} \ell(x_k, y_k, u_k) &
	\label{eq:mpc_general_formal:cost}\\
	\text{s.t.} \ & x_{k+1} = f(x_k, u_k, d_k),  & k \in \mathbb{N}_{0}^{N-1} & \label{eq:mpc_general_formal:xp} \\
	& y_{k} = g(x_k, u_k, d_k),  & k \in \mathbb{N}_{0}^{N-1} & \label{eq:mpc_general_formal:yp} \\
	&  x_{k} \in \mathcal{X},  & k \in \mathbb{N}_{0}^{N-1}   \label{eq:mpc_general_formal:ub}\\
	& u_{k} \in \mathcal{U}, & k \in \mathbb{N}_{0}^{N-1} 
	\label{eq:mpc_general_formal:xb}\\
	%   & \underline{x} \le x_k \le \overline{x}, \label{eq:mpc_general_formal:xb}\\
	%   & \underline{u} \le u_k \le \overline{u}, \label{eq:mpc_general_formal:ub}\\
	& x_0 = x(t),\label{eq:mpc_general_formal:x0} \\
	& d_0 = d(t),\label{eq:mpc_general_formal:d0}
	\end{align}
\end{subequations}
where $x_k \in \mathbb{R}^{n_x}$, $y_k \in \mathbb{R}^{n_y}$, $u_k \in \mathbb{R}^{n_u}$ and $d_k \in \mathbb{R}^{n_d}$ denote the values of states, outputs,
inputs and disturbances, respectively, at the $k$-th step of the prediction
horizon $N$. 
Objective function is given by~\eqref{eq:mpc_general_formal:cost} where   $\ell_N(x_N)$  represents the terminal penalty,
while $\ell(x_k,u_k)$  is called a stage cost and its
purpose is to assign a cost to a particular choice of $x_k$ and
$u_k$.
The predictions of the state values are obtained from the state update equation~\eqref{eq:mpc_general_formal:xp}, while values of the predicted outputs are given by
the output equation~\eqref{eq:mpc_general_formal:yp}.
States and inputs are subject to
constraints~\eqref{eq:mpc_general_formal:ub} and~\eqref{eq:mpc_general_formal:xb}.
Initial conditions of the problem are given by~\eqref{eq:mpc_general_formal:x0} for the state variables which are either  measured or estimated, and by~\eqref{eq:mpc_general_formal:d0} for the disturbances which are measured for the current timestep and forecasted for the future.
For generality we can denote by $\xi$ the vector that encapsulates all time-varying parameters
of~\eqref{eq:mpc_general_formal}, i.e. the current states $x(t)$,
current and future disturbances $d(t), \ldots, d(t+N T_s)$, or possibly other parameters such as comfort boundaries or reference signals which depend on specific formulation.

\section{MPC FORMULATION IN BUILDINGS: GRANTING A PHYSICAL MEANING }

Now that the MPC abstract formulation has been defined, 
physical meaning and ranges of variables have to be specified.
Table~\ref{tab:mpc_form:translation} summarizes the most common
metrics found in building control and maps them with the abstract
notation from the control engineering notation. 


\begin{table}[h]
	% Suppressing floating placement of tables/figures in LaTeX is generally deprecated. 
	\centering
	\caption{MPC formulation notation translation}
	\label{tab:mpc_form:translation}
	\begin{tabular}{l|c|l|cccc}
		\toprule
		\multicolumn{3}{l}{\textbf{Physical domain}} & \multicolumn{4}{r}{\textbf{Abstract domain}} \\
		\toprule
		\textbf{Variables} & \textbf{Symbol} & \textbf{Description} & \textbf{$x$} & \textbf{$y$} & \textbf{$u$} & \textbf{$d$} \\ 
		\midrule
		\multirow{5}{*}{\textbf{Temperatures}} & $T$ & Envelope temperatures (wall, concrete, glazing...) & $\bullet$ & - & - & - \\ 
		& $T_z$ & Zone operative temperatures & - & $\bullet$ & - & - \\
		& $T_{sup}$ & Air/water supply temperatures &  - & - & $\bullet$ & - \\
		& $T_e$ & Ambient temperature &  - & - & - & $\bullet$ \\
		& $T_b$ & Borefield deep temperatures & $\bullet$ & - & - & - \\
		\midrule
		\multirow{5}{*}{\textbf{Heat flows}} &
		$Q_{HVAC}$ & Total heat flow of HVAC components & - & - & $\bullet$ & - \\
		& $Q_{i}$ & Heat flow of the HVAC component $i$ & - & - & - & $\bullet$ \\
		& $Q_{rad}$ & Solar radiation & - & - & - & $\bullet$ \\
		& $Q_{occ}$ & Occupancy internal gains & - & - & - & $\bullet$ \\
		& $Q_{lig}$ & Lighting internal gains & - & - & $\bullet$ & $\bullet$ \\
		\midrule
		\multirow{5}{*}{\textbf{Component signals}} &
		$T_{set}$ & Supply system set-points & - & - & $\bullet$ & - \\
		& $x_{pump}$ & Pump/fan speed signal & - & - & $\bullet$ & - \\
		& $\Delta p$ & Pump/fan dif. pressure & - & - & $\bullet$ & - \\
		& $x_{val}$ & Valve positions & - & - & $\bullet$ & - \\
		& $x_{dam}$ & Damper positions & - & - & $\bullet$ & - \\
		\midrule
		\multirow{4}{*}{\textbf{IEQ}} &
		$\dot{m}_{air}$ & Air mass flow & - & - & $\bullet$ & - \\
		& $PMV$ & PMV & - & $\bullet$ & - & - \\
		& $\phi$ & Humidity & - & $\bullet$ & - & $\bullet$ \\
		& $I_v$ & Illuminance & - & $\bullet$ & - & - \\
		\bottomrule 
	\end{tabular}
\end{table} 

The main objective of building control is the maximization of 
the occupant comfort while using as less resources as possible. 
This definition leads to a multi-objective optimization problem,
with two conflicting terms: an energy-related term and a comfort
related term. The balance between these terms is typically adjusted
by means of weighting terms to give priority to the different targets.
In human perspective, these weighting factors represent the "price"
that the user is willing to pay to have more or less comfort.
Eq.~\eqref{eq:mpc_general_formal:cost}
can be re-written as a multi-objective function~\eqref{eq:mpc_building_formal:cost},
where $\ell_c$ represents an arbitrary weighted discomfort term and $\ell_e$
stands for the weighted energetic term. 

Table~\ref{tab:mpc_form:objectives} recaps the most common objectives used 
in the MPC formulation. The energetic term accounts for the energy used in all
the HVAC components of the building. For $n$ HVAC components in the building, 
the energy used in the building stands for the sum of the energy of each
HVAC component $i$, which can be calculated as a function $\alpha_i$ of the states,
inputs and disturbances found in our model. Generally, the energy used by
any HVAC component $i$ can be calculated as the ratio between the heat that
delivers $Q_i$ and its efficiency $\eta_i$. 

In function of which objective it is desired to minimised, these $\alpha_i$
functions can be transformed using a pricing factor $P_i$, which can be added
as a forecasted disturbance to the formulation. Energy can be converted to
different metrics depending on their nature. Monetary cost is determined 
by using a cost factor $p_i$ that can be variable (electricity price),
constant (fuel price), negative (PV injection to the grid)... $CO_2$ footprint
is calculated with an emission factor $e_i$ and the share of renewable energy
sources (RES) can be computed with the share percentage of each component $r_i$.
 
Both terms are weighted with the matrices $Q_e$ and $Q_c$
in order to adjust the trade-off between terms.

\renewcommand{\arraystretch}{1.5}
\begin{table}[h]
	% Suppressing floating placement of tables/figures in LaTeX is generally deprecated. 
	\centering
	\caption{MPC formulation objectives}
	\label{tab:mpc_form:objectives}
	\begin{tabular}{l|c|c}
		\toprule
		\multicolumn{3}{l}{\textbf{Energy-related objective term $\ell_e = Q_e \sum_{i=1}^{n} P_i \alpha_i(x_k, y_k, d_k)$}} \\
		\midrule
		\textbf{Linguistic objective} & \textbf{Pricing Factor} & \textbf{MPC formulation} \\
		\midrule
		General form & $P_i$ &  $\alpha_i(x_k, u_k, d_k)$ \\
		\midrule
		Minimise energy use & $P_i = 1$ & \multirow{4}{*}{ $\dfrac{Q_{i}}{\eta_i}$} \\
		Minimise cost & $P_i = p_i(t)$ &  \\
		Minimise $CO_2$ emissions & $P_i = e_i(t)$ & \\
		Maximise RES & $P_i = 1 - r_i(t)$ &   \\
		\bottomrule 
	\end{tabular}
\end{table}

In general, there are two types of constraints,
inequality  (control inputs range, comfort zones, etc.) and equality 
(building model dynamics, rate limits, etc.) constraints.
The constraints which satisfaction is mandatory are called 
\textit{hard constraints}. For example, control actions
bounds~\eqref{eq:mpc_building_formal:ub} limit the formulation
according to the limitations (e.g., in power) of the HVAC components
and they need to be satysfied at every time instant for whole prediction horizon.

On the other hand, the constraints which can be violated are
known as \textit{soft constraints}. They are usually relaxed
by additional slack variables $s_k$ that are added to and
penalized in the objective function~\eqref{eq:mpc_building_formal:cost}.
Soft constraints commonly used in building control (e.g. thermal comfort
zone inequality constraints) are given by eq.~\eqref{eq:mpc_building_formal:yb},
defined by upper $\overline{y}_k$  and lower bounds $\underline{y}_k$
on controlled variable $y_{k}$. These constrainst can be time-varying.

From practical perspective in building control applications,
the constraints are most commonly used to enforce selected
variables to stay within given ranges, e.g., heat influxes
and room temperatures, supply air temperatures, airflow rates,
and others HVAC variables, or for tuning the MPC performance via,
e.g., slew rate constraints on input variables.


\begin{subequations}
	\label{eq:mpc_building_formal}
	\begin{align}
	\min_{u_0, \ldots, u_{N-1}}  
	&\sum_{k=0}^{N-1} \ell_c(x_k,u_k,d_k) + \ell_e(x_k,u_k,d_k) &
	\label{eq:mpc_building_formal:cost}\\
	\text{s.t.} \ & x_{k+1} = f(x_k, u_k, d_k),  & k \in \mathbb{N}_{0}^{N-1} & \label{eq:mpc_general_formal:xp} \\
	& y_{k} = g(x_k, u_k, d_k),  & k \in \mathbb{N}_{0}^{N-1} & \label{eq:mpc_general_formal:yp} \\
	& \underline{u}_k   \le u_{k} \le  \overline{u}_k    \label{eq:mpc_building_formal:ub}\\
	& \underline{y}_k  -s_k \le y_{k} \le  \overline{y}_k + s_k  
	\label{eq:mpc_building_formal:yb}\\
	%   & \underline{x} \le x_k \le \overline{x}, \label{eq:mpc_general_formal:xb}\\
	%   & \underline{u} \le u_k \le \overline{u}, \label{eq:mpc_general_formal:ub}\\
	& x_0 = x(t),\label{eq:mpc_general_formal:x0} \\
	& d_0 = d(t),\label{eq:mpc_general_formal:d0}
	\end{align}
\end{subequations}






\bibliographystyle{apacite}
\bibliography{template}
\vspace{24pt}


\section*{ACKNOWLEDGMENTS}

This work emerged from the IBPSA Project 1, an international project conducted under the umbrella of the International Building Performance Simulation Association (IBPSA). Project 1 will develop and demonstrate a BIM/GIS and Modelica Framework for building and community energy system design and operation.

\end{document}
