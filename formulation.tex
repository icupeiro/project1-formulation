% This conference manuscript template is prepared for:   
% Kim Stockment, Conference Coordinator, Ray W. Herrick Laboratories, Purdue University, West Lafayette, IN, USA. 
% Latest revision = 2016-02-23 

\documentclass[10pt]{extarticle}
% Miktex users: require l3 packages
%             : optional 3 packages
\usepackage{amssymb,amsmath,multicol,titlesec,apacite,booktabs,tabto,url, multirow, makecell, eurosym}
%\usepackage[round]{natbib}  

\usepackage[hmargin = 1in, vmargin = 1in]{geometry}

\renewcommand{\refname}{References}
    
%\usepackage[MnSymbol]{mathspec}
%\setallmainfonts{Times New Roman}
% Mathspec requires the XeTeX compiler. 

\usepackage[parfill]{parskip}
\usepackage{graphicx}
\usepackage[font=bf]{caption}

\usepackage{titlesec} 
%\titleformat{command}[shape]{format}{label}{sep}{before}[after]
\titleformat{\section}[hang]{\fontsize{12pt}{1em}\selectfont \bfseries}{\thesection. }{0pt}{\centering \MakeUppercase}
\titleformat{\subsection}[hang]{\fontsize{11pt}{1em}\selectfont \bfseries}{\thesubsection}{5pt}{}
\titleformat{\subsubsection}[runin]{}{\thesubsubsection}{5pt}{} 
%\titlespacing{command}{left}{beforesep}{aftersep}[right]
\titlespacing{\section}{0pt}{10pt}{10pt}
\titlespacing{\subsection}{0pt}{10pt}{0pt}
\titlespacing{\subsubsection}{0pt}{10pt}{0pt}	
%\setlength{\bibsep}{3pt}
%\pagenumbering{\gobble}
%\setlength{\hyphenpenalty}{1000}
%\setlength{\exhyphenpenalty}{1000}

\usepackage{fancyhdr}
\pagestyle{fancy}
\fancyhead[R]{\fontsize{12pt}{1em}\selectfont {{\textbf{Page \thepage}}}}
\fancyhead[L]{}
\fancyfoot[C]{IBPSA Project 1, September 2018}
\renewcommand{\headrulewidth}{0pt} % Turn off the bar

\newcommand\thefontsize[1]{{#1 The current font size is: \f@size pt\par}}

\begin{document}
	
\begin{center}
\vspace{0.2in}
\noindent{\fontsize{14pt}{1em}\selectfont \textbf{MPC formulation description template}}\\[14pt]

{\fontsize{11pt}{1.2em}\selectfont
 J\'an Drgo\v na\textsuperscript{1}, Iago Cupeiro Figueroa\textsuperscript{2}
\\[11pt]

KU Leuven\\
Department of Mechanical Engineering - TME Division\\
The Sysi's \\[11pt]

jan.drgona@kuleuven.be\textsuperscript{1},
iago.cupeirofigueroa@kuleuven.be\textsuperscript{2} \\
}
\end{center}

\vspace{0.5cm}


\section*{ABSTRACT}

This document contains the MPC formulation template within the frame of the IBPSA Project 1.
The proposed template aims to adapt all possible formulations for the different types of models and key performance indicators (KPIs) in consideration.
It has to be general.
To this end, we choose the control engineering notation as our starting point,
since it considers the formulation in a general way from an abstract point of view.
From here, we  move to a formulation with physical meanind based on the specific characteristics of the building sector.

%\section*{NOMENCLATURE}
%
%\begin{samepage}
%	$x$\tabto{1.0in}Vector of states	\\
%	$u$\tabto{1.0in}Vector of inputs	\\
%	$y$\tabto{1.0in}Vector of outputs\\	
%	$d$\tabto{1.0in}Vector of disturbances\ \\
%	$N$\tabto{1.0in}Prediction horizon
%	
%\end{samepage}
%\begin{samepage}
%	\textbf{Subscript}\\
%	$k$\tabto{1.0in}Considered time-step index
%\end{samepage}


\section{CONTROL ENGINEERING NOTATION}

General MPC formulation can be cast as a following optimal control problem (OCP) in discrete time
\begin{subequations}
	\label{eq:mpc_general_formal}
	\begin{align}
	\min_{u_0, \ldots, u_{N-1}} & \ell_N(x_N) + \sum_{k=0}^{N-1} \ell(x_k, y_k, u_k, r_k) &
	\label{eq:mpc_general_formal:cost}\\
	\text{s.t.} \ & x_{k+1} = f(x_k, u_k, d_k),  & k \in \mathbb{N}_{0}^{N-1} & \label{eq:mpc_general_formal:xp} \\
	& y_{k} = g(x_k, u_k, d_k),  & k \in \mathbb{N}_{0}^{N-1} & \label{eq:mpc_general_formal:yp} \\
	&  x_{k} \in \mathcal{X},  & k \in \mathbb{N}_{0}^{N-1}   \label{eq:mpc_general_formal:ub}\\
	& u_{k} \in \mathcal{U}, & k \in \mathbb{N}_{0}^{N-1} 
	\label{eq:mpc_general_formal:xb}\\
	%   & \underline{x} \le x_k \le \overline{x}, \label{eq:mpc_general_formal:xb}\\
	%   & \underline{u} \le u_k \le \overline{u}, \label{eq:mpc_general_formal:ub}\\
	& x_0 = x(t),\label{eq:mpc_general_formal:x0} \\
	& d_0 = d(t),\label{eq:mpc_general_formal:d0}
	\end{align}
\end{subequations}
where $x_k \in \mathbb{R}^{n_x}$, $y_k \in \mathbb{R}^{n_y}$, $u_k \in \mathbb{R}^{n_u}$, $d_k \in \mathbb{R}^{n_d}$ and $r_k \in \mathbb{R}^{n_r}$ denote the values of states, outputs,
inputs, disturbances and reference signals, respectively, at the $k$-th step of the prediction horizon $N$. 
Objective function is given by~\eqref{eq:mpc_general_formal:cost} where   $\ell_N(x_N)$  represents the terminal penalty,
while $\ell(x_k,u_k)$  is called a stage cost and its
purpose is to assign a cost to a particular choice of $x_k$ and
$u_k$.
The predictions of the state values are obtained from the state update equation~\eqref{eq:mpc_general_formal:xp}, while values of the predicted outputs are given by
the output equation~\eqref{eq:mpc_general_formal:yp}.
States and inputs are subject to
constraints~\eqref{eq:mpc_general_formal:ub} and~\eqref{eq:mpc_general_formal:xb}.
Initial conditions of the problem are given by~\eqref{eq:mpc_general_formal:x0} for the state variables which are either  measured or estimated, and by~\eqref{eq:mpc_general_formal:d0} for the disturbances which are measured for the current timestep and forecasted for the future.
For generality we can denote by $\xi$ the vector that encapsulates all time-varying parameters
of~\eqref{eq:mpc_general_formal}, i.e. the current states $x(t)$,
current and future disturbances $d(t), \ldots, d(t+N T_s)$, or possibly other parameters such as comfort boundaries or reference signals which depend on specific formulation.

\section{MPC FORMULATION IN BUILDINGS: GRANTING A PHYSICAL MEANING }

Now that the MPC abstract formulation has been defined, 
physical meaning and ranges of variables have to be specified.
Table~\ref{tab:mpc_form:translation} summarizes the most common
variables used in building control and maps them with the abstract
notation from the control engineering notation. 
because these variables are usually vectors, we will indicate  the
$i$-th element of the vector variable via supercsript $i$.
For instance the operative temperature in $i$-th zone will be denoted as $T_z^i$.

An overview of the most common parameters in building control together with their associated varialbes  is given in Table~\ref{tab:mpc_form:parameters}.
Please note that these parameters can be treated either as a constant numerial values or time-varying parameters. In the latter case the variables can be technically labeled as a reference signals $r$ within MPC formulation.

Table~\ref{tab:mpc_form:objectives} recaps the most common objectives used 
in the MPC formulation. These objectives can be combined in a multi-objective
function. Since the main objective of building control is the maximization of 
the occupant comfort while using as less resources as possible,
usually two conflicting terms are found: an energy-related term
and a comfort related term.   
 
The energy used by the building stands for the sum of the energy of each
HVAC component $Q^i_{\text{HVAC}}$.
Generally, the power $P^i_{\text{HVAC}}$ of $i$-th HVAC component can be calculated as the
ratio between the heat that it delivers $Q^i_{\text{HVAC}}$ and its efficiency $COP^i, EER^i$ or $\eta^i$. 
These efficiencies are parameters that can be considered either constant or variable
in function of the inputs, states and disturbances.
The energy use
minimisation function can be further transformed to monetary cost  using a price factor  $p^i$, which can be added to the formulation 
as variable (in electric-based components), constant (in fuel-based components) or even negative 
(renewable-based components that pour energy into the grid) parameter.
The emission factor $e^i$ converts the energy into the associatted $CO_2$ emissions 
for the $i$-th component  and it can be variable on electric-based components,
constant for fuel-based components and zero on renewable-based components.
The renewable energy sources (RES) share $r^i$ represents the fraction of renewable
energy coming from the  $i$-th component and depends on the share of the grid for
electric-based components, for fuel-based ones the share is zero and for
renewable-based components the share would be unitary.
The discomfort is typically evaluated via slack variable $s_k$ which denotes violation of associated comfort  zone for a particular physical variable, e.g. zone operative temperature $T_z$.
The balance between the different objectives is typically adjusted
by means of weighting terms $Q_{xx}$ to give more or less priority to the associated objective term $\ell_{xx}$.
For example the penalization on control inputs can be expressed as $Q_{u}$.

In general, there are two types of constraints,
inequality  (control inputs range, comfort zones, etc.) and equality 
(building model dynamics, rate limits, etc.) constraints.
A recap of the most common constraints in building control can be found at
Table \ref{tab:mpc_form:constraints}.
The constraints which satisfaction is mandatory are called 
\textit{hard constraints}. For example, hard control actions
bounds define the minumum $\underline{u}_k$ and maximum $\overline{u}_k$ allowed values fo the contol variables.
 On the other hand, the constraints which can 
be violated are known as \textit{soft constraints}. They are usually relaxed
by additional slack variables $s_k$ that are added to and
penalized in the objective function (see Comfort objective in Table
\ref{tab:mpc_form:objectives}), trying to keep within the established
comfort bounds. The constrainst can be time-varying.

% TODO: Q_i is conflicting notation with weights, keep using Q_{HVAC}^i
% TODO: we could add units
\begin{table}[h]
	% Suppressing floating placement of tables/figures in LaTeX is generally deprecated. 
	\centering
	\caption{MPC variables notation and translation between physical and abstract domain.}
	\label{tab:mpc_form:translation}
	\begin{tabular}{l|c|l|cccc}
		\toprule
		\multicolumn{3}{l}{\textbf{Physical domain}} &  \multicolumn{4}{r}{\textbf{Abstract domain}} \\
		\toprule
		\textbf{Variables} & \textbf{Symbol} & \textbf{Description} & \textbf{$x$} & \textbf{$y$} & \textbf{$u$} & \textbf{$d$}  \\ 
		\midrule
		\multirow{8}{*}{\textbf{Temperatures}} & $T$ & Envelope temperatures (wall, concrete, glazing...) & $\bullet$ & -  & - & - \\ 
		& $T_{\text{z}}$ & Zone perative temperatures  & - & $\bullet$ & -  & - \\
		& $T_{\text{sup}}$ & Air/water supply temperatures &  - & - & $\bullet$ & -  \\
		& $T_{\text{ret}}$ & Air/water return temperature of the component $i$ &  - & - & $\bullet$ & - \\
		& $T_\text{e}$ & Ambient temperature &  - & - & - & $\bullet$ \\
		& $T_\text{b}$ & Borefield wall temperature & $\bullet$ & - & - & - \\
			& $T_\text{g}$ & Borefield undisturbed ground temperature & $\bullet$ & - & - & - \\
			& $T_{\text{sg}i}$ & Borefield surrounding ground temperature at distance $i$ & $\bullet$ & - & - & - \\
		\midrule
		\multirow{5}{*}{\textbf{Heat flows}} 
% 		& $Q$ & Total heat flow  of HVAC components & - & - & $\bullet$ & -  \\
		& $Q_{\text{HVAC}}$ & Heat flow of  HVAC components  & - & - &  $\bullet$ &- \\
		& $Q_{\text{rad}}$ & Solar radiation & - & - & - & $\bullet$  \\
		& $Q_{\text{occ}}$ & Occupancy internal gains & - & - & - & $\bullet$  \\
		& $Q_{\text{lig}}$ & Lighting internal gains & - & - & $\bullet$ & $\bullet$  \\
		\midrule
		\multirow{3}{*}{\textbf{Power used}} 
% 	& $P$ & Total power  of HVAC components & - & - & $\bullet$ & - \\
		& $P_{\text{HVAC}}$ & Power of HVAC components & - & - &  $\bullet$ &- \\
		& $P_{\text{lig}}$ & Lighting power & - & - & $\bullet$ & $\bullet$ \\
\midrule
		\multirow{2}{*}{\textbf{Mass flows}} &
		$m_{\text{wat}}$ & Water mass flow & - & - & $\bullet$ & - \\
		& $m_{\text{air}}$ & Air mass flow & - & - & $\bullet$ & - \\
		\midrule
		\multirow{4}{*}{\textbf{\shortstack[l]{Component signals }}} 
% 		& $T^{set}_{\text{sup}}$ & Supply system set-points & - & - & $\bullet$ & - \\
% 		& $T^{set}_{z}$ & Zone set-point & - & - & $\bullet$ & - \\
		& $\Delta p$ & Pump/fan dif. pressure & - & - & $\bullet$ & - \\
		& $x_{\text{mov}}$ & Pump/fan speed signal & - & - & $\bullet$ & - \\
		& $x_{\text{val}}$ & Valve positions & - & - & $\bullet$ & - \\
		& $x_{\text{dam}}$ & Damper positions & - & - & $\bullet$ & - \\
		\midrule
		\multirow{4}{*}{\textbf{\shortstack[l]{Indoor Environmental \\ Quality (IEQ)}}} &
		$PMV$ & Predicted mean vote & - & $\bullet$ & - & - \\
		& $CO_2$ & $CO_2$ concentration in air & - & $\bullet$ & - & $\bullet$ \\
		& $\phi$ & Humidity & - & $\bullet$ & - & $\bullet$ \\
		& $E_{\text{v}}$ & Illuminance & - & $\bullet$ & - & $\bullet$ \\
		\bottomrule 
	\end{tabular}
\end{table} 


\renewcommand{\arraystretch}{2}
\begin{table}[h]
	% Suppressing floating placement of tables/figures in LaTeX is generally deprecated. 
	\centering
	\caption{MPC formulation parameters and reference signals.}
	\label{tab:mpc_form:parameters}
	\begin{tabular}{l|c|l|c}
		\toprule
		\textbf{Energy efficiencies}  & \textbf{Symbol} &  \textbf{Description} & \textbf{Associated variables} \\
		\midrule
		\makecell[l]{Coefficient of \\ performance} & $COP^i$ &  \makecell[l]{Heating efficiency of heat pump \\ or air conditioning system $i$} & $T_{sup}, Q^i_{\text{HVAC}}, P^i_{\text{HVAC}}$ \\
		\makecell[l]{Energy efficiency \\ ratio} & $EER^i$ & \makecell[l]{Cooling efficiency of heat pump \\ or air conditioning system $i$} & $T_{sup}, Q^i_{\text{HVAC}}, P^i_{\text{HVAC}}$ \\
		Efficiency & $\eta$ & \makecell[l]{Efficiency of all other systems} & $T_{sup}, Q^i_{\text{HVAC}}, P^i_{\text{HVAC}}$ \\
		\midrule
		\textbf{Other energy factors}  & \textbf{Symbol} &  \textbf{Description} \\
		\midrule
		Price factor & $p^i$ &  \makecell[l]{Conversion factor from energy \\ to monetary cost of component $i$} & $Q^i_{\text{HVAC}}$ \\
		Emission factor & $e^i$ & \makecell[l]{Conversion factor from energy \\ to $CO_2$ emissions of component $i$} & $Q^i_{\text{HVAC}}$ \\
		RES factor & $r^i$ & \makecell[l]{Ratio of renewable energy of \\ the considered component $i$} & $Q^i_{\text{HVAC}}$ \\
		\midrule
		\textbf{Comfort bounds}  & \textbf{Symbol} &  \textbf{Description} \\
		\midrule
		Temperature & $\underline{T_{\text{z}}},\overline{T_{\text{z}}}$ & \makecell[l]{Zone operative temperature  \\ comfort bounds } & $T_{\text{z}}$ \\
		$CO_2$ concentration & $\underline{CO_2},\overline{CO_2}$ & \makecell[l]{$CO_2$ concentration comfort \\ bounds} & $CO_{2}$ \\
		Predicted mean vote & $\underline{PMV},\overline{PMV}$ &  \makecell[l]{Predicted mean vote comfort \\ bounds} & $PMV$ \\
		Humidity & $\underline{\phi},\overline{\phi}$ &  Humidity comfort bounds & $\phi$  \\
		Illuminance & $\underline{E_{\text{v}}},\overline{E_{\text{v}}}$ & Illuminance comfort bounds & $E_{\text{v}}$  \\
		\bottomrule 
		\textbf{Component limitations}  & \textbf{Symbol} &  \textbf{Description} \\
		\midrule
		Heat flow limit & $\underline{Q^i_{\text{HVAC}}},\overline{Q^i_{\text{HVAC}}}$ & \makecell[l]{Minimum/maximum heat flow of \\ the $i$-th HVAC component} & $Q^i_{\text{HVAC}}$ \\
		Heat flow rate of change & $\underline{\Delta Q^i_{\text{HVAC}}},\overline{\Delta Q^i_{\text{HVAC}}}$ & \makecell[l]{Minimum/maximum heat flow rate of \\ change of the  $i$-th HVAC component} & $Q^i_{\text{HVAC}}$ \\
		\midrule
		\textbf{Auxiliary parameters}  & \textbf{Symbol} &  \textbf{Description} \\
		\midrule
		\makecell[l]{Arbitrary weighting \\ term} & $Q_{\text{xx}}$ &  \makecell[l]{Weighting for the particular term \\ in the objective function} & $x,y,u,d,r$ \\
		\makecell[l]{Slack variable} & $s$ &  \makecell[l]{Used to soften the constraints, usually \\ for the given comfort requirement} 
		&  $T_{z}, CO_{2}, PMV, \phi, E_{\text{v}}$\\
		\makecell[l]{Time-step} & $t_s$ &  \makecell[l]{Time-step used in the \\ optimization problem} & $x,y,u,d,r$  \\
		\bottomrule 
	\end{tabular}
\end{table}

% TODO: add definition and use of mathematical norms, 1,2,inf
\renewcommand{\arraystretch}{2.5}
\begin{table}[h]
	% Suppressing floating placement of tables/figures in LaTeX is generally deprecated. 
	\centering
	\caption{Individual terms of MPC objective $\ell(x_k, y_k, u_k, d_k, r_k)$.}
	\label{tab:mpc_form:objectives}
	\begin{tabular}{l|c}
		\toprule
		\textbf{Linguistic objective}  & \textbf{MPC formulation} \\
		\midrule
		Minimise energy use &   $  \sum_{i=1}^{n_u} \dfrac{Q^{i}_{\text{HVAC}}}{\eta^i}$ \\
		Minimise cost & $  \sum_{i=1}^{n_u} p^i \dfrac{Q^{i}_{\text{HVAC}}}{\eta^i}$  \\
		Minimise $CO_2$ emissions & $ \sum_{i=1}^{n_u}  e^i \dfrac{Q^{i}_{\text{HVAC}}}{\eta^i}$  \\
		Maximise RES &  $ \sum_{i=1}^{n_u} (1-r^i) \dfrac{Q^{i}_{\text{HVAC}}}{\eta_i}$  \\
		Flexibility & ???  \\
		Minimize discomfort &  $ \sum_{i=1}^{n_y} ||s^i||^2_2$ \\
		\bottomrule 
	\end{tabular}
\end{table}

\renewcommand{\arraystretch}{2.5}
\begin{table}[h]
	% Suppressing floating placement of tables/figures in LaTeX is generally deprecated. 
	\centering
	\caption{MPC formulation constraints.}
	\label{tab:mpc_form:constraints}
	\begin{tabular}{l|lll}
		\toprule
		\textbf{Linguistic constraint}  &  \multicolumn{3}{c}{\textbf{MPC formulation}} \\
		& Class &  Control notation &  Physical notation \\
		\midrule
		Model dynamics & hard equality & $ x_{k+1} = f(x_k,u_k,d_k) $  & $ T_{k+1} = f(T_k,Q_{\text{HVAC},k},T_{\text{e},k},Q_{\text{rad},k},Q_{\text{occ},k},Q_{\text{lig},k}) $ \\ 
		Component limitations & hard inequality & $ \underline{u}_k \le u_k \le \overline{u}_k  $  & $ \underline{Q_{\text{HVAC}}}_k  \le Q_{\text{HVAC},k}  \le \overline{Q_{\text{HVAC}}}_k  $ \\ 
		Component difference & hard equality & $\Delta u_k = u_{k-1} - u_k $  & $\Delta Q_{\text{HVAC},k} = Q_{\text{HVAC},k-1} - Q_{\text{HVAC},k} $  \\ 
		Component rate of change & hard inequality & $ \underline{\Delta u}_k   \le \Delta u_k  \le \overline{\Delta u}_k  $ & $ \underline{\Delta Q_{\text{HVAC}}}_k   \le \Delta Q_{\text{HVAC},k}  \le \overline{\Delta Q_{\text{HVAC}}}_k  $ \\ 
		Temperature comfort zone & soft inequality & $ \underline{y}_k - s_k \le y_k \le \overline{y}_k + s_k $  & $ \underline{T_{\text{z}}}_k - s^{T_{\text{z}}}_{k} \le T_{\text{z}} \le \overline{T_{\text{z}}}_k + s^{T_{\text{z}}}_{k} $ \\
		$CO_2$ concentration & soft inequality & $ \underline{y}_k - s_k \le y_k \le \overline{y}_k + s_k $ & $ \underline{{CO_2}}_k - s^{CO_2}_k \le CO_2 \le \overline{{CO_2}}_k + s^{CO_2}_k $  \\
		Predicted mean vote &  soft inequality & $ \underline{y}_k - s_k \le y_k \le \overline{y}_k + s_k $ & $ \underline{PMV}_k - s^{PMV}_k \le PMV \le \overline{PMV}_k + s^{PMV}_k $  \\
		Humidity & soft inequality & $ \underline{y}_k - s_k \le y_k \le \overline{y}_k + s_k $ & $ \underline{\phi}_k - s^{\phi}_k \le \phi \le \overline{\phi}_k + s^{\phi}_k $  \\
		Illuminance & soft inequality & $ \underline{y}_k - s_k \le y_k \le \overline{y}_k + s_k $ & $ \underline{{E_\text{v}}}_k - s^{E_\text{v}}_k \le E_\text{v} \le \overline{{E_\text{v}}}_k + s^{E_\text{v}}_k $  \\
		\bottomrule 
	\end{tabular}
\end{table}


\bibliographystyle{apacite}
\bibliography{template}
\vspace{24pt}


\section*{ACKNOWLEDGMENTS}

This work emerged from the IBPSA Project 1, an international project conducted under the umbrella of the International Building Performance Simulation Association (IBPSA). Project 1 will develop and demonstrate a BIM/GIS and Modelica Framework for building and community energy system design and operation.

The authors acknowledge the financial support by the European Union through  the EU-H2020-GEOT\euro CH 
project ‘Geothermal Technology for \euro conomic Cooling and Heating’ 
and within the H2020-EE-2016-RIA-IAprogramme for the project ‘Model Predictive Control and Innovative System Integration of GEOTABS;-) 
in Hybrid Low Grade Thermal Energy Systems - Hybrid MPC GEOTABS’ (grant number 723649 - MPC-; GT). 

\end{document}
