% This conference manuscript template is prepared for:   
% Kim Stockment, Conference Coordinator, Ray W. Herrick Laboratories, Purdue University, West Lafayette, IN, USA. 
% Latest revision = 2016-02-23 

\documentclass[10pt]{extarticle}
% Miktex users: require l3 packages
%             : optional 3 packages
\usepackage{amssymb,amsmath,multicol,titlesec,apacite,booktabs,tabto,url, multirow}
%\usepackage[round]{natbib}  

\usepackage[hmargin = 1in, vmargin = 1in]{geometry}

\renewcommand{\refname}{References}
    
%\usepackage[MnSymbol]{mathspec}
%\setallmainfonts{Times New Roman}
% Mathspec requires the XeTeX compiler. 

\usepackage[parfill]{parskip}
\usepackage{graphicx}
\usepackage[font=bf]{caption}

\usepackage{titlesec} 
%\titleformat{command}[shape]{format}{label}{sep}{before}[after]
\titleformat{\section}[hang]{\fontsize{12pt}{1em}\selectfont \bfseries}{\thesection. }{0pt}{\centering \MakeUppercase}
\titleformat{\subsection}[hang]{\fontsize{11pt}{1em}\selectfont \bfseries}{\thesubsection}{5pt}{}
\titleformat{\subsubsection}[runin]{}{\thesubsubsection}{5pt}{} 
%\titlespacing{command}{left}{beforesep}{aftersep}[right]
\titlespacing{\section}{0pt}{10pt}{10pt}
\titlespacing{\subsection}{0pt}{10pt}{0pt}
\titlespacing{\subsubsection}{0pt}{10pt}{0pt}	
%\setlength{\bibsep}{3pt}
%\pagenumbering{\gobble}
%\setlength{\hyphenpenalty}{1000}
%\setlength{\exhyphenpenalty}{1000}

\usepackage{fancyhdr}
\pagestyle{fancy}
\fancyhead[R]{\fontsize{12pt}{1em}\selectfont {{\textbf{Page \thepage}}}}
\fancyhead[L]{}
\fancyfoot[C]{IBPSA Project 1, September 2018}
\renewcommand{\headrulewidth}{0pt} % Turn off the bar

\newcommand\thefontsize[1]{{#1 The current font size is: \f@size pt\par}}

\begin{document}
	
\begin{center}
\vspace{0.2in}
\noindent{\fontsize{14pt}{1em}\selectfont \textbf{MPC formulation description template}}\\[14pt]

{\fontsize{11pt}{1.2em}\selectfont
Iago Cupeiro Figueroa\textsuperscript{1}, J\'an Drgo\v na\textsuperscript{2}
\\[11pt]

KU Leuven\\
Department of Mechanical Engineering - TME Division\\
The Sysi's \\[11pt]

iago.cupeirofigueroa@kuleuven.be\textsuperscript{1}, jan.drgona@kuleuven.be\textsuperscript{2} \\
}
\end{center}

\vspace{0.5cm}


\section*{ABSTRACT}

This document contains the MPC formulation template within the frame of the IBPSA Project 1.
The proposed template has to adapt all possible formulations for the different types of model and key performance indicators (KPIs) in consideration.
It has to be general.
To this end, we choose the control engineering notation as our starting point,
since it considers the formulation in a general way from an abstract point of view.
From here, we aim to move to a more physical formulation based on the particularities and specific characteristics of the building sector.

\section*{NOMENCLATURE}

\begin{samepage}
	$x$\tabto{1.0in}Vector of states	\\
	$u$\tabto{1.0in}Vector of inputs	\\
	$y$\tabto{1.0in}Vector of outputs\\	
	$d$\tabto{1.0in}Vector of disturbances\ \\
	$N$\tabto{1.0in}Prediction horizon
	
\end{samepage}
\begin{samepage}
	\textbf{Subscript}\\
	$k$\tabto{1.0in}Considered time-step index
\end{samepage}


\section{CONTROL ENGINEERING NOTATION}

General MPC formulation can be cast as a following optimal control problem (OCP) in discrete time
\begin{subequations}
	\label{eq:mpc_general_formal}
	\begin{align}
	\min_{u_0, \ldots, u_{N-1}} & \ell_N(x_N) + \sum_{k=0}^{N-1} \ell(x_k, y_k, u_k) &
	\label{eq:mpc_general_formal:cost}\\
	\text{s.t.} \ & x_{k+1} = f(x_k, u_k, d_k),  & k \in \mathbb{N}_{0}^{N-1} & \label{eq:mpc_general_formal:xp} \\
	& y_{k} = g(x_k, u_k, d_k),  & k \in \mathbb{N}_{0}^{N-1} & \label{eq:mpc_general_formal:yp} \\
	&  x_{k} \in \mathcal{X},  & k \in \mathbb{N}_{0}^{N-1}   \label{eq:mpc_general_formal:ub}\\
	& u_{k} \in \mathcal{U}, & k \in \mathbb{N}_{0}^{N-1} 
	\label{eq:mpc_general_formal:xb}\\
	%   & \underline{x} \le x_k \le \overline{x}, \label{eq:mpc_general_formal:xb}\\
	%   & \underline{u} \le u_k \le \overline{u}, \label{eq:mpc_general_formal:ub}\\
	& x_0 = x(t),\label{eq:mpc_general_formal:x0} \\
	& d_0 = d(t),\label{eq:mpc_general_formal:d0}
	\end{align}
\end{subequations}
where $x_k \in \mathbb{R}^{n_x}$, $y_k \in \mathbb{R}^{n_y}$, $u_k \in \mathbb{R}^{n_u}$ and $d_k \in \mathbb{R}^{n_d}$ denote the values of states, outputs,
inputs and disturbances, respectively, at the $k$-th step of the prediction
horizon $N$. 
Objective function is given by~\eqref{eq:mpc_general_formal:cost} where   $\ell_N(x_N)$  represents the terminal penalty,
while $\ell(x_k,u_k)$  is called a stage cost and its
purpose is to assign a cost to a particular choice of $x_k$ and
$u_k$.
The predictions of the state values are obtained from the state update equation~\eqref{eq:mpc_general_formal:xp}, while values of the predicted outputs are given by
the output equation~\eqref{eq:mpc_general_formal:yp}.
States and inputs are subject to
constraints~\eqref{eq:mpc_general_formal:ub} and~\eqref{eq:mpc_general_formal:xb}.
Initial conditions of the problem are given by~\eqref{eq:mpc_general_formal:x0} for the state variables which are either  measured or estimated, and by~\eqref{eq:mpc_general_formal:d0} for the disturbances which are measured for the current timestep and forecasted for the future.
For genrality we can denote by $\xi$ the vector that encapsulates all time-varying parameters
of~\eqref{eq:mpc_general_formal}, i.e. the current states $x(t)$,
current and future disturbances $d(t), \ldots, d(t+N T_s)$, or possibly other parameters such as comfort boundaries or reference signals which depend on specific formulation.

\section{MPC FORMULATION IN BUILDINGS: GRANTING A PHYSICAL MEANING }

Now that the MPC abstract formulation has been defined, 
physical meaning and ranges of variables have to be specified.

\begin{table}[b]
% Suppressing floating placement of tables/figures in LaTeX is generally deprecated. 
\centering
\caption{MPC formulation translation}
\begin{tabular}{l|l|cccc}
\toprule
\multicolumn{2}{r}{}  & \multicolumn{4}{r}{\textbf{Abstract domain}} \\
\toprule
\multicolumn{2}{l}{\textbf{Physical domain}} & \textbf{$x$} & \textbf{$y$} & \textbf{$u$} & \textbf{$d$} \\ 
\midrule
\multirow{5}{*}{\textbf{Temperatures}} & \textbf{Envelope temperatures (wall, concrete, glazing...)} & $\bullet$ & - & - & - \\ 
& \textbf{Zone operative temperatures} & - & $\bullet$ & - & - \\
& \textbf{Air/water supply temperatures} & - & - & $\bullet$ & - \\
& \textbf{Ambient temperature} & - & - & - & $\bullet$ \\
& \textbf{Borefield deep temperatures} & $\bullet$ & - & - & - \\
\midrule
\multirow{4}{*}{\textbf{Heat sources}} &
\textbf{HVAC components} & - & - & $\bullet$ & - \\
& \textbf{Solar radiation} & - & - & - & $\bullet$ \\
& \textbf{Occupants} & - & - & - & $\bullet$ \\
& \textbf{Lighting} & - & - & $\bullet$ & $\bullet$ \\
\midrule
\multirow{4}{*}{\textbf{Component signals}} &
\textbf{Supply system set-points} & - & - & $\bullet$ & - \\
& \textbf{Pump/fan speeds/dif. pressure} & - & - & $\bullet$ & - \\
& \textbf{Valve positions} & - & - & $\bullet$ & - \\
& \textbf{Damper positions} & - & - & $\bullet$ & - \\
\midrule
\multirow{4}{*}{\textbf{IEQ}} &
 \textbf{Air mass flow} & - & - & $\bullet$ & - \\
& \textbf{PMV} & - & $\bullet$ & - & - \\
& \textbf{Humidity} & - & $\bullet$ & - & $\bullet$ \\
& \textbf{Illuminance} & - & $\bullet$ & - & - \\
\bottomrule 
\end{tabular}
\end{table} 




\bibliographystyle{apacite}
\bibliography{template}
\vspace{24pt}
\emph{References} are to be placed at the end of the manuscript in alphabetical order by the last name of the first author.  All sources cited in the text have to be listed in the list of references in alphabetical order of the author's name or of the lead author's name if there are several authors at the end of the manuscript.  The sources should be listed in the standard APA (American Psychological Association) style.  The \texttt{apacite} package will format references in the APA style. 

Examples of APA style citations for other print and electronic sources can be found on the Purdue OWL website, \url{https://owl.english.purdue.edu/owl/resource/560/1/}.

The American Psychological Association maintains a blog where detailed instructions, commentary, and examples for citing different sources are provided: \url{http://blog.apastyle.org/apastyle/}.

Standard abbreviations for periodicals and journals can be found on Web of Science maintained by Thomson Reuters: \url{https://images.webofknowledge.com/WOK46/help/WOS/A_abrvjt.html}.  

\section*{ACKNOWLEDGMENTS}

An acknowledgment can be located at the end of the text to indicate the sponsor of the study presented and/or to acknowledge additional contributors to the paper. 

\end{document}
